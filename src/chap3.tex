\chapter{原子磁矩}
物质磁性:原(离)子磁性基础。
具有复合性:
\begin{enumerate}
    \item 原子局域环境(晶胞形状) 
    \item 集体行为(晶格振动) 
\end{enumerate}


原子核与核外电子之间:
\begin{enumerate}
    \item 量子处理
    \item 壳层结构
    \item 电子-电子,电子-核相互作用
\end{enumerate}
%——————————————————————————————————%
\section{电子磁矩和磁性}



\subsection{电子磁矩}
电子角动量:自旋 + 轨道
\subsubsection{轨道磁矩}
\begin{enumerate}
\item 经典:运动轨迹

若电子半径为$r$,为圆形,角速率$\omega$:

磁矩:
\begin{align*}
    \vec{m} &= \mu_0 I \vec{S}\\
    m &= \mu_0 (\frac{-e\omega}{2\pi})\pi r^2\\
    &= \frac{1}{2} \mu_0 e \omega r^2\\
\end{align*}

轨道角动量:
\begin{align*}
    \vec{l} &= m_e \vec{r} \times \vec{v}\\
    l &= m_e \omega r^2\\
    \vec{m} &= -\frac{\mu_0 e}{2m_e}\vec{l}
\end{align*}
\item 量子:具有特定轨道角动量的状态

对于:
\[ \hat{m} = -\frac{\mu_0 e}{2 m_e}\hat{m} \]
仅:
\[ \hat{m_e} = -\frac{\mu_0 e}{2 m_e}\hat{m_z} \]
\[ m_{\text{eff}} = \sqrt{l(l+1)}\hbar \]
实验可测:
\begin{align*}
    m_z &= -\frac{\mu_0 e}{2m_e}m_l \hbar\\
    &= -m_l\mu_B
\end{align*}
玻尔磁子:
\[\mu_B \equiv \frac{\mu_0 e \hbar}{2m_e}\]
\end{enumerate}
\subsubsection{自旋磁矩}
自旋角动量:
\begin{align*}
    \hat{s}^2 &\to s(s+1)\hbar^2, &s=\frac{1}{2}\\
    \hat{s}_z &\to m_s \hbar,   &m_s=\pm \frac{1}{2}
\end{align*}

有:
\begin{align*}
    m_z^s &= -2m_s\mu_B\\
    m_{\text{eff}}^s &= \sqrt{s(s+1)}\hbar^2
\end{align*}

\subsubsection{总磁矩}
\begin{align*}
    m_z &= -m_j g \mu_B\\
    \vec{j} &= \vec{l} + \vec{s}\\
    \vec{j}_z &= m_j \hbar
\end{align*}

当$\vec{j} = \vec{l}$,无自旋,$g=1$,$m_z=-m_l\mu_B$;

当$\vec{j} = \vec{s}$,无轨道,$g=2$,$m_z=-2m_s\mu_B$。

\begin{enumerate}
    \item 朗德因子$g$:
    \begin{align*}
        g &= \frac{\text{测量到的}m_z/\mu_B}{\text{角动量投影}/\hbar}\\
        g &= \frac{\text{总磁矩}\vec{j}\text{方向投影}/\mu_B}
        {\text{总角动量}/\hbar}\\
        &= \frac{\vec{m}_\text{总}\vec{j}/\mu_B}
        {|\vec{j}|^2/\hbar}
    \end{align*}

    引入:
    \begin{align*}
        \hat{m}_\text{总}&=-\frac{\hat{l}}{\hbar}\mu_B
        -2\frac{\hat{s}}{\hbar}\mu_B\\
        &= -\frac{\mu_B}{\hbar}(\hat{l}+2\hat{s})
    \end{align*}
    \begin{align*}
        \vec{m}_\text{总}\vec{j}
        &= -\frac{\mu_B}{\hbar}
            [(\hat{l}+2\hat{s})(\hat{l}+\hat{s})]\\
        &= -\frac{\mu_B}{\hbar}
            (\hat{l}^2+3\hat{l}\cdot\hat{s}+2\hat{s}^2)\\
        &= -\frac{\mu_B}{\hbar}
            (\frac{2}{3}\hat{j}^2-\h\hat{l}^2+\h\hat{s}^2)\\
    \end{align*}

    代入本征值得:
    \[ g = \frac{\frac{2}{3}j(j+1)-\h l(l+1)+\h s(s+1)}{j(j+1)} \]

    \item 旋磁比:
    
    电子磁矩与角动量的比值:
    \[\gamma = \frac{m}{j} = g\frac{\mu_B}{\hbar}
    =g \frac{\mu_0e}{2m_e} \]
    \[ \vec{m} = -\gamma j \]
    $g$即为无量纲的$\gamma$。
\end{enumerate}

\subsection{电子磁矩的动力学方程}



\subsection{电子的磁性}



%——————————————————————————————————%
\section{原子中局域电子的磁性}